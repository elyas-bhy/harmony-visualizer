\abstract{\emph{Data mining in software repositories has a lot of potential for discovering important information about the development process of a project. However, this data is not always usable in it's brute form, and it needs a visualisation in order to become apparent. Different visualisations are adapted to making different information apparent. There are several metrics used in software repository visualisation. We chose workload distribution in order to determine how developers interact with one another during a projects life-cycle. Therefore we visualise contributions to a repository from two points of view, the developers', and the components'. This gives us a vision of the components a developer has worked on, as well as the amount of work he has performed on said component. It also gives us a vision of the work received by a component, thus potentially detecting collaborations, or even determining ownership of said component. This is all information which can be useful to a team of developers in order to optimise some of the processes they employ, however, we can't be sure that our presentation of the information is more efficient than just the brute data without performing an in-depth study on the matter.}}