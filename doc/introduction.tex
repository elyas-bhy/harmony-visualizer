\section{Introduction}

As students in software engineering, specialized in project management, we are quite interested by all the information that can be extracted from a source code repository. Indeed, that data can easily enlighten the manager about the project state, by laying bare the development process. 

In the field of software engineering and data mining, researchers use code metrics to analyze, evaluate and improve software quality. The IEEE Standard Glossary of Software Engineering Terms defines \emph{metric} as \emph{"a quantitative measure of the degree to which a system, component, or process possesses a given attribute"}\cite{RSPressman}.\\
After acquiring more knowledge about ownership\cite{Girba2005,Girba2007} and focus metrics\cite{Posnett}, we also understood that we could extract even more information than raw data, depending on how we structure the visualisation of this data.

Thus, once we have explained the main points of the articles we were interested in and exposing what knowledge they bring, we will describe the visualisation we chose to develop, Harmony Visualiser, which shows the workload distribution in a project, we will then expose the protocol and the results of an empirical study conducted in order to measure the efficiency of this visualisation.