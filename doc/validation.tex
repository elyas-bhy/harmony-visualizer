\section{Validation}

\subsection{Survey Context}

As a step towards validating the efficiency of our visualisation, we established an empirical study to measure its performance.
We chose to compare our visualisation with raw data (in tabular format) in order to obtain a first impression of the limitations of our tool when compared to the most basic form of data presentation.
For this analysis, we chose two Apache projects: cordova-plugin-contacts (project A) and cordova-plugin-file-transfer (project B). Larger projects will rapidly become difficult to read data in tabular format, thus potentially biasing the survey in favor of our tool, which would not respect the concept of an empirical study. In order to conduct a thorough study, we would need to conduct the survey on several projects of varying sizes. However we have used our best judgement to choose two projects which will hopefully best resemble the results of a more thorough study.

In this survey, we will be interrogating people with an IT background (namely developers), on visualisations of open-source projects.

During the course of this survey, we will be taking into consideration the level of experience and status of the subjects that are answering the questions, in order to establish whether or not there exists a correlation between the level of expertise of a developer, and his/her ability to correctly answer our survey.


\subsection{Experimental Protocol}

In order to underline an eventual difference in efficiency between our visualization and a tabular view, we need to compare the time efficiency of both views on different test subjects. Thus, one test group will be provided with the visualisation of project A and the tabular equivalent for project B. The second test group will be provided with the opposite combination.

To measure the efficiency of both views, we will simply record the time required by the participant to complete the survey. In order to do so, participants will be required to fill an online form\footnote{\url{http://ryan.herbert.emi.u-bordeaux1.fr/harmony-visualizer/source/website/web/}}, and will be provided with either of the two combinations mentioned above.

In order to compare the efficiency of a view compared to another, we need to calculate several performance indicators. For this, we use boxplots to display differences in performance through their quartiles. We can then determine whether our visualization tool is useful for a large number of people or not, and the difference in efficiency of a view compared to the other.

\subsection{Survey}

The following survey is provided to the participants of this study:

\textbf{Tell us about yourself:}
\begin{enumerate}
\itemsep-0.2em
\item What is your job description?
\item What is your level of experience in the world of software development (in terms of years)?
\item Have you ever used a visualization tool in a similar context?
\item Do you have a visual deficiency (colour blindness for instance)?
\item If you have contributed to an Apache project, please indicate your nickname.
\item If you wish to know the results of this survey, please provide us with your email address.
\end{enumerate}

\textbf{About the project:}
\begin{enumerate}
\itemsep-0.2em
\item Find the most active developer (in terms of contributions).
\item Find the developer who contributed to the highest number of components.
\item Find the component which has the highest amount of developers.
\item Find the component which received the highest amount of contributions.
\item Find the component which received the highest amount of contributions by \emph{purplecabbage}.
\item Find the most active developer on the root (“/”) component.
\item Find a component to which one developer contributed much more than the other developers.
\item Indicate the total number of contributions of \emph{Anis Kadri}.
\item On a scale of 1 to 5, indicate whether the workload is equally divided amongst developers (5/5 being perfectly equal).
\item On a scale of 1 to 5, does it seems that developers work on the same components (5/5), or is there only one developer per component (1/5) ?
\item On a scale of 1 to 5, rate the development team’s composition: is it composed of isolated groups which tend to work together (1/5), or is the team one large group in which all members seem to work together (5/5) ?
\item According to you, which developer has the most knowledge on the project (i.e. the one which contributed the most to the project globally)?
\end{enumerate}